\documentclass[a4paper, titlepage]{article}

\usepackage[utf8]{inputenc} % accents
\usepackage[T1]{fontenc}      % caractères français
\usepackage{geometry}         % marges
\usepackage[frenchb]{babel}  % langue
\usepackage{graphicx}         % images
\usepackage{verbatim}         % texte préformaté
\usepackage{hhline}
\usepackage{flafter} 
\usepackage[section]{placeins}
\usepackage{listings}
\lstset{language=Python }

\title{Rapport projet Combinatoire}      
\author{-}         
\date{-}          

                              
\begin{document}
%% \maketitle

\section{Réponses attendues pour la méthode count}
Pour la grammaire des arbres :
\begin{center}
\begin{tabular}{|c||c|c|c|c|c|c|c|c|c|c|c|}
\hline n & 0 & 1 & 2 & 3 & 4 & 5 & 6 & 7 & 8 & 9 & 10 \\
\hline
\hline count(n, Tree) & 0 & 1 & 1 & 2 & 5 & 14 & 42 & 123 & 429 & 1430
& 4862 \\
\hline count(n, Node) & 0 & 0 & 1 & 2 & 5 & 14 & 42 & 123 & 429 & 1430 & 4862 \\
\hline count(n, Leaf) & 0 & 1 & 0 & 0 & 0 & 0 & 0 & 0 & 0 & 0 & 0\\
\hline
\end{tabular}
\end{center}

Pour la grammaire des mots de Fibonacci :
\begin{center}
\begin{tabular}{|l||c|c|c|c|c|c|c|c|c|c|c|}
\hline n & 0 & 1 & 2 & 3 & 4 & 5 & 6 & 7 & 8 & 9 & 10 \\
\hline
\hline count(n, Fib) & 1 & 2 & 3 & 5 & 8 & 13 & 21 & 34 & 55 & 89 & 144 \\
\hline count(n, Cas1) & 0 & 2 & 3 & 5 & 8 & 13 & 21 & 34 & 55 & 89 & 144 \\
\hline count(n, Cas2) & 0 & 1 & 1 & 2 & 3 & 5 & 8 & 13 & 21 & 34 & 55\\
\hline count(n, Vide) & 1 & 0 & 0 & 0 & 0 & 0 & 0 & 0 & 0 & 0 & 0\\
\hline count(n, CasAu) & 0 & 1 & 2 & 3 & 5 & 8 & 13 & 21 & 34 & 55 & 89\\
\hline count(n, AtomA) & 0 & 1 & 0 & 0 & 0 & 0 & 0 & 0 & 0 & 0 & 0\\
\hline count(n, AtomB) & 0 & 1 & 0 & 0 & 0 & 0 & 0 & 0 & 0 & 0 & 0\\
\hline count(n, CasBAu) & 0 & 0 & 1 & 2 & 3 & 5 & 8 & 13 & 21 & 34 & 55\\
\hline
\end{tabular}
\end{center}

\section{Grammaires}

\subsection{Alphabet A,B}
\medskip
$ S\ =\ \mathcal{E}\ |\ AS\ |\ BS $ 

$w$ est un mot de la grammaire $A$, $B$ si:
\begin{itemize}
\item soit $w$ est vide
\item soit $w$ est de la forme $Au$ où $u$ est un mot de la grammaire
  $A,B$
\item soit $w$ est de la forme $Bu$ où $u$ est un mot de la grammaire
  $A,B$
\end{itemize}

\subsection{Mots de Dyck}
\medskip
$ S\ =\ \mathcal{E}\ |\ (S)\ |\ S(S) $ 

$w$ est un mot de Dyck si:
\begin{itemize}
\item soit $w$ est vide
\item soit $w$ est de la forme $(u)$ où $u$ est un mot de Dyck
  de Dyck
\item soit $w$ est de la forme $u(v)$ où $u$ et $v$ sont des mots de Dyck
\end{itemize}

\subsection{Mots sur l'alphabet A,B qui n'ont pas trois lettres
  consécutives égales}
\medskip
$ S\ =\ \mathcal{E}\ |\ U\ |\ T $

$ U\ =\ A\ |\ AA\ |\ AT\ |\ AAT$

$ T\ =\ B\ |\ BB\ |\ BT\ |\ BBT$

$w$ est un mot de cette grammaire si:
\begin{itemize}
\item soit $w$ est vide
\item soit $w$ est de la forme $A$, $AA$, $B$, $BB$
\item soit $w$ est de la forme $AT$ ou $AAT$ où $T$ est un mot de la grammaire
  qui commence par $B$
\item soit $w$ est de la forme $BU$ ou $BBU$ où $U$ est un mot de la grammaire
  qui commence par $A$

\end{itemize}

\subsection{Palindromes sur l'alphabet A, B}
\medskip
$ S\ =\ \mathcal{E}\ |\ A\ |\ B\ |\ ASA\ |\ BSB $

$w$ est un mot de la grammaire des palindrome sur $A$, $B$ si:
\begin{itemize}
\item soit $w$ est vide
\item soit $w$ est de la forme $A$ ou $B$
\item soit $w$ est de la forme $AuA$ où $u$ est un palindrome.
\item soit $w$ est de la forme $BuB$ où $u$ est un palindrome.
\end{itemize}

\subsection{Palindromes sur l'alphabet A, B, C}
\medskip
$ S\ =\ \mathcal{E}\ |\ A\ |\ B\ |\ ASA\ |\ BSB\ |\ CSC$

$w$ est un mot de la grammaire des palindrome sur $A$, $B$ si:
\begin{itemize}
\item soit $w$ est vide
\item soit $w$ est de la forme $A$ ou $B$ ou $C$
\item soit $w$ est de la forme $AuA$ où $u$ est un palindrome.
\item soit $w$ est de la forme $BuB$ où $u$ est un palindrome.
\item soit $w$ est de la forme $CuC$ où $u$ est un palindrome.
\end{itemize}

\subsection{Mots sur l'alphabet A,B qui contiennent autant de A que de
B}
\medskip
$ S\ =\ \mathcal{E}\ |\ aTbS\ |\ bUaS$

$ T\ =\ \mathcal{E}\ |\ aTbT$

$ U\ =\ \mathcal{E}\ |\ bTaT$


\section{Calcul de la valuation}

\subsection{Mots de Fibonacci}
\begin{table}[!hbt]
\centering
\small
\setlength\tabcolsep{2pt}
\begin{tabular}{|c|cccccccc|}
\hline $n$ & Fib & Cas1 & Cas2 & Vide & CasAu & AtomA & AtomB & CasBAu\\
\hline
\hline règle & Vide $\cup$ Cas1 & CasAU $\cup$ Cas2 & AtomA $\cup$
AtomB & $\mathcal{E}$ & AtomA*Fib & A & B & AtomB*CasAu\\ 
\hline
\hline
0 & $\infty$ &  $\infty$ & $\infty$ & $\infty$ & $\infty$ & $\infty$ &$\infty$ & $\infty$ \\
1 & $\infty$ &  $\infty$ & $\infty$ & 0 & $\infty$ & 1 & 1 & $\infty$ \\
2 & 0 &  $\infty$ & 1 & 0 & 1 & 1 & 1 & $\infty$ \\
3 & 0 & 1 & 1 & 0 & 1 & 1 & 1 & 2 \\
3 & 0 & 1 & 1 & 0 & 1 & 1 & 1 & 2\\
\hline
\end{tabular}
\end{table}

\newpage
\subsection{Mots de Dyck}
\begin{table}[!hbt]
\centering
\small
\setlength\tabcolsep{2pt}
\begin{tabular}{|c|ccccccc|}
\hline $n$ & Dyck & Casuu & Vide & AtomLPAR & AtomRPAR & Cas(u & Casu) \\
\hline
\hline règle & Vide $\cup$ Casuu & Dyck * Cas(u & $\matchcal{E}$
&``(`` & ``)'' $ & LPAR * Casu) & Dyck * RPAR \\ 
\hline
\hline
0 & $\infty$ &  $\infty$ & $\infty$ & $\infty$ & $\infty$ & $\infty$ &$\infty$ \\
1 & $\infty$ &  $\infty$ & 0 & 1 & 1 & $\infty$ &$\infty$ \\
2 & 0 &  $\infty$ & 0 & 1 & 1 & $\infty$ &$\infty$ \\
3 & 0 &  $\infty$ & 0 & 1 & 1 & $\infty$ & 1 \\
4 & 0 &  $\infty$ & 0 & 1 & 1 & 2 & 1 \\
5 & 0 & 2 & 0 & 1 & 1 & 2 & 1 \\
6 & 0 & 2 & 0 & 1 & 1 & 2 & 1 \\

\hline

\end{tabular}
\end{table}




\begin{table}[!hbt]
\subsection{Mots sur l'alphabet A,B}
\centering
\small
\setlength\tabcolsep{2pt}
\begin{tabular}{|c|ccccccc|}
\hline $n$ & AB & AtomA & AtomB & CasAB & Vide & CasAu & CasBu \\
\hline
\hline règle & Vide $\cup$ CasAB & A & B & CasAu $\cup$ CasBu &
$\mathcal{E}$ & AtomA * AB & AtomB * AB \\
\hline
\hline
0 & $\infty$ &  $\infty$ & $\infty$ & $\infty$ & $\infty$ & $\infty$ &$\infty$ \\
1 & $\infty$ &  1 & 1 & $\infty$ & 0 & $\infty$ &$\infty$ \\
2 & 0 &  1 & 1 & $\infty$ & 0 & 1 & 1 \\
3 & 0 &  1 & 1 & 1 & 0 & 1 & 1 \\
4 & 0 &  1 & 1 & 1 & 0 & 1 & 1 \\


\hline

\end{tabular}
\end{table}

\begin{table}[!hbt]
\subsection{Mots qui n'ont pas trois lettres consécutives égales sur A, B}
\centering
\small
\setlength\tabcolsep{2pt}
\begin{tabular}{|c|ccccccccc|}
\hline $n$ & Three & Vide & AtomA & AtomB & AA & BB & S & U & U1 \\
\hline
\hline règle & Vide $\cup$ S & $\mathcal{E}$ & A & B & AtomA*AtomA &
AtomB*AtomB & U $\cup$ T & AtomA $\cup$ U1 & AA $\cup$ U2\\
\hline
\hline
0 & $\infty$ &  $\infty$ & $\infty$ & $\infty$ & $\infty$ & $\infty$ &
$\infty$ & $\infty$ & $\infty$ \\
1 & $\infty$ & 0  & 1 & 1 & $\infty$ & $\infty$ &
$\infty$ & $\infty$ & $\infty$ \\
2 & 0 & 0  & 1 & 1 & 2 & 2 &
$\infty$ & 1 & $\infty$ \\
3 & 0 & 0  & 1 & 1 & 2 & 2 &
1& 1 & 2 \\
4 & 0 & 0  & 1 & 1 & 2 & 2 &
1& 1 & 2 \\
5 & 0 & 0  & 1 & 1 & 2 & 2 &
1& 1 & 2 \\
6 & 0 & 0  & 1 & 1 & 2 & 2 &
1& 1 & 2 \\

\hline
\end{tabular}

\vspace{1cm}
\begin{tabular}{|c|cccccccc|}
\hline $n$ & U2 & AT & AAT & T & T1 & T2 & BU & BBU \\
\hline
\hline règle & AT $\cup$ AAT & AtomA * T & AtomA * AT & AtomB $\cup$
T1 & BB $\cup$ T2 & BU $\cup$ BBU & AtomB * U & AtomB * BU\\
\hline
0 & $\infty$ &  $\infty$ & $\infty$ & $\infty$ & $\infty$ & $\infty$ &
$\infty$ & $\infty$  \\
1 & $\infty$ &  $\infty$ & $\infty$ & $\infty$ & $\infty$ & $\infty$ &
$\infty$ & $\infty$  \\
2 & $\infty$ &  $\infty$ & $\infty$ & 1 & $\infty$ & $\infty$ &
$\infty$ & $\infty$  \\
3 & $\infty$ & 2 & $\infty$ & 1 & 2 & $\infty$ &
2 & $\infty$  \\
4 & 2 & 2 & 3 & 1 & 2 & $\infty$ &
2 & 3  \\
5 & 2 & 2 & 3 & 1 & 2 & 2 &
2 & 3  \\
6 & 2 & 2 & 3 & 1 & 2 & 2 &
2 & 3  \\

\hline
\end {tabular}
\end{table}

\newpage
\begin{table}[!hbt]
\subsection{Palindromes sur A,B}
\centering
\small
\setlength\tabcolsep{2pt}
\begin{tabular}{|c|cccccc|}
\hline $n$ & Pal & Vide & AtomA & AtomB & S & S1\\
\hline
\hline règle & Vide $\cup$ S & $\mathcal{E}$ & A & B & AtomA $\cup$ S1
& AtomB $\cup$ S2 \\
\hline
\hline
0 & $\infty$ &  $\infty$ & $\infty$ & $\infty$ & $\infty$ & $\infty$\\
1 & $\infty$ & 0 & 1 & 1 & $\infty$ & $\infty$\\
2 & 0 & 0 & 1 & 1 & 1 & 1\\
3 & 0 & 0 & 1 & 1 & 1 & 1\\
4 & 0 & 0 & 1 & 1 & 1 & 1\\
5 & 0 & 0 & 1 & 1 & 1 & 1\\
6 & 0 & 0 & 1 & 1 & 1 & 1\\

\hline
\end{tabular}

\vspace{1cm}

\begin{tabular}{|c|ccccc|}
\hline $n$ & S2 & ASA & ASA1 & BSB & BSB1  \\
\hline
\hline règle & ASA $\cup$ BSB & AtomA * ASA1 & Pal * AtomA &
AtomB * BSB1 & Pal * AtomB\\
\hline
\hline
0 & $\infty$ &  $\infty$ & $\infty$ & $\infty$ & $\infty$ \\
1 & $\infty$ &  $\infty$ & $\infty$ & $\infty$ & $\infty$ \\
2 & $\infty$ &  $\infty$ & $\infty$ & $\infty$ & $\infty$ \\
3 & $\infty$ &  $\infty$ & 1 & $\infty$ & 1 \\
4 & $\infty$ &  2 & 1 & 2 & 1 \\
5 & 2 &  2 & 1 & 2 & 1 \\
6 & 2 &  2 & 1 & 2 & 1 \\

\hline
\end{tabular}
\end{table}

\begin{table}[!hbt]
\subsection{Palindromes sur A,B,C}
\centering
\small
\setlength\tabcolsep{2pt}
\begin{tabular}{|c|ccccccccc|}
\hline $n$ & Pal & Vide & AtomA & AtomB & AtomC & S & S1 & S2 & S3\\
\hline
\hline règle & Vide $\cup$ S & $\mathcal{E}$ & A & B & C & AtomA $\cup$ S1
& AtomB $\cup$ S2 & AtomC $\cup$ S3 & ASA $\cup$ S4 \\
\hline
\hline
0 & $\infty$ &  $\infty$ & $\infty$ & $\infty$ & $\infty$ & $\infty$ & $\infty$ & $\infty$ & $\infty$ \\
1 & $\infty$ & 0 & 1 & 1 & 1 & $\infty$ & $\infty$ & $\infty$ & $\infty$ \\
2 & 0 & 0 & 1 & 1 & 1 & 1 & 1 & 1 & $\infty$ \\
3 & 0 & 0 & 1 & 1 & 1 & 1 & 1 & 1 & $\infty$ \\
4 & 0 & 0 & 1 & 1 & 1 & 1 & 1 & 1 & $\infty$ \\
5 & 0 & 0 & 1 & 1 & 1 & 1 & 1 & 1 & 2 \\
6 & 0 & 0 & 1 & 1 & 1 & 1 & 1 & 1 & 2 \\

\hline
\end{tabular}


\vspace{1cm}

\begin{tabular}{|c|ccccccc|}
\hline $n$  & S4 & ASA & ASA1 & BSB & BSB1 & CSC & CSC1  \\
\hline
\hline règle & BSB $\cup$ CSC & AtomA * ASA1 & Pal * AtomA &
AtomB * BSB1 & Pal * AtomB & AtomC * CSC1 & Pal * AtomC\\
\hline
\hline
0 & $\infty$ &  $\infty$ & $\infty$ & $\infty$ & $\infty$ & $\infty$ & $\infty$ \\
1 & $\infty$ &  $\infty$ & $\infty$ & $\infty$ & $\infty$ & $\infty$ & $\infty$ \\
2 & $\infty$ &  $\infty$ & $\infty$ & $\infty$ & $\infty$ & $\infty$ & $\infty$ \\
3 & $\infty$ &  $\infty$ & 1 & $\infty$ & 1 & $\infty$ & 1 \\
4 & $\infty$ &  2 & 1 & 2 & 1 & 2 & 1 \\
5 & 2 & 2 & 1 & 2 & 1 & 2 & 1 \\
6 & 2 & 2 & 1 & 2 & 1 & 2 & 1 \\

\hline
\end{tabular}


\end{table}

\newpage

\subsection{Palindromes sur A,B,C}

\begin{table}[!hbt]
\centering
\small
\setlength\tabcolsep{2pt}
\begin{tabular}{|c|ccccccccc|}
\hline $n$ & Pal & Vide & AtomA & AtomB & AtomC & S & S1 & S2 & S3\\
\hline
\hline règle & Vide $\cup$ S & $\mathcal{E}$ & A & B & C & AtomA $\cup$ S1
& AtomB $\cup$ S2 & AtomC $\cup$ S3 & ASA $\cup$ S4 \\
\hline
\hline
0 & $\infty$ &  $\infty$ & $\infty$ & $\infty$ & $\infty$ & $\infty$ & $\infty$ & $\infty$ & $\infty$ \\
1 & $\infty$ & 0 & 1 & 1 & 1 & $\infty$ & $\infty$ & $\infty$ & $\infty$ \\
2 & 0 & 0 & 1 & 1 & 1 & 1 & 1 & 1 & $\infty$ \\
3 & 0 & 0 & 1 & 1 & 1 & 1 & 1 & 1 & $\infty$ \\
4 & 0 & 0 & 1 & 1 & 1 & 1 & 1 & 1 & $\infty$ \\
5 & 0 & 0 & 1 & 1 & 1 & 1 & 1 & 1 & 2 \\
6 & 0 & 0 & 1 & 1 & 1 & 1 & 1 & 1 & 2 \\

\hline
\end{tabular}

\vspace{1cm}

\begin{tabular}{|c|ccccccc|}
\hline $n$  & S4 & ASA & ASA1 & BSB & BSB1 & CSC & CSC1  \\
\hline
\hline règle & BSB $\cup$ CSC & AtomA * ASA1 & Pal * AtomA &
AtomB * BSB1 & Pal * AtomB & AtomC * CSC1 & Pal * AtomC\\
\hline
\hline
0 & $\infty$ &  $\infty$ & $\infty$ & $\infty$ & $\infty$ & $\infty$ & $\infty$ \\
1 & $\infty$ &  $\infty$ & $\infty$ & $\infty$ & $\infty$ & $\infty$ & $\infty$ \\
2 & $\infty$ &  $\infty$ & $\infty$ & $\infty$ & $\infty$ & $\infty$ & $\infty$ \\
3 & $\infty$ &  $\infty$ & 1 & $\infty$ & 1 & $\infty$ & 1 \\
4 & $\infty$ &  2 & 1 & 2 & 1 & 2 & 1 \\
5 & 2 & 2 & 1 & 2 & 1 & 2 & 1 \\
6 & 2 & 2 & 1 & 2 & 1 & 2 & 1 \\

\hline
\end{tabular}
\end{table}

\subsection{Mots sur A,B qui contiennent autant de A que de B}

\begin{table}[!hbt]
\centering
\small
\setlength\tabcolsep{2pt}
\begin{tabular}{|c|cccccccccc|}
\hline $n$ & Vide & AtomA & AtomB & S & S1 & T & U & aTbS & TbS & bS\\
\hline
\hline règle & $\mathcal{E}$ & A & B &  $\mathcal{E}$ $\cup$ S1 &
aTbS $\cup$ bUaS &  $\mathcal{E}$ $\cup$ aTbT &  $\mathcal{E}$ $\cup$
bUaU & A * TbS & T * bS & B * S  \\
\hline
\hline
0 & $\infty$ &  $\infty$ & $\infty$ & $\infty$ & $\infty$ & $\infty$ &
$\infty$ & $\infty$ & $\infty$  & $\infty$  \\
1 & 0 & 1 & 1 & $\infty$ & $\infty$ & $\infty$ &
$\infty$ & $\infty$ & $\infty$  & $\infty$  \\
2 & 0 & 1 & 1 & 0 & $\infty$ & 0 &
0 & $\infty$ & $\infty$  & $\infty$  \\
3 & 0 & 1 & 1 & 0 & $\infty$ & 0 & 0 & $\infty$ & $\infty$  & 1  \\
4 & 0 & 1 & 1 & 0 & $\infty$ & 0 & 0 & $\infty$ & 1 & 1  \\
5 & 0 & 1 & 1 & 0 & $\infty$ & 0 & 0 & 2 & 1 & 1  \\
6 & 0 & 1 & 1 & 0 & 2 & 0 & 0 & 2 & 1 & 1  \\
7 & 0 & 1 & 1 & 0 & 2 & 0 & 0 & 2 & 1 & 1  \\

\hline
\end{tabular}

\vspace{1cm}

\begin{tabular}{|c|ccccccccc|}
\hline $n$ & bUaS & UaS & aS & aTbT & TbT & bT & bUaU & UaU & aU \\
\hline
\hline règle & B * UaS  & U * aS & A * S & A * TbT & T * bT & B * T & B * UaU & U
* aU & A * U\\
\hline
\hline
0 & $\infty$ &  $\infty$ & $\infty$ & $\infty$ & $\infty$ & $\infty$ &
$\infty$  & $\infty$ & $\infty$ \\
1 & $\infty$ &  $\infty$ & $\infty$ & $\infty$ & $\infty$ & $\infty$ &
$\infty$  & $\infty$ & $\infty$ \\
2 & $\infty$ &  $\infty$ & $\infty$ & $\infty$ & $\infty$ & $\infty$ &
$\infty$  & $\infty$ & $\infty$ \\
3 & $\infty$ &  $\infty$ & 1 & $\infty$ & $\infty$ & 1 &
$\infty$  & $\infty$ & 1 \\
4 & $\infty$ & 1 & 1 & $\infty$ & 1  & 1 &
$\infty$  & 1 & 1 \\
5 & 2 & 1 & 1 & 2 & 1 & 1 & 2  & 1 & 1 \\
6 & 2 & 1 & 1 & 2 & 1 & 1 & 2  & 1 & 1 \\
7 & 2 & 1 & 1 & 2 & 1 & 1 & 2  & 1 & 1 \\

\hline
\end{tabular}
\end{table}


\section{Fonction qui vérifie qu'une grammaire est correcte}

Pour vérifier qu'une grammaire est correcte, on parcourt toutes les
règles de la grammaire et pour chaque règle, s'il s'agit d'une règle
de type ConstructorRule, on vérifie si les deux règles qu'elle possède
en attributs sont définies dans la grammaire.

ADD STYLE

\begin{lstlisting}
  def checkDefinedRules (grammar):
     for ruleId in grammar:
     rule = grammar[ruleId]
     if isinstance(rule, R.ConstructorRule):
     fst, snd = rule.parameters
     if not (fst in grammar):
     raise IncorrectGrammar(fst+" rule not defined in grammar")
     if not (snd in grammar):
     raise IncorrectGrammar(snd+" rule not defined in grammar")
\end{lstlisting}

\section {Structure du programme}

\section{Tests de cohérence génériques}

Les propriétés suivantes doivent etres vérifiées par les grammaires :\\
\begin{itemize}
\item Pour toute règle r d'une grammaire, pour tout entier n positif ou
  nul, \\ r.count(n) == len(r.list(n))
\item ([rule.rank(i) for i in rule.list(n)] ==
  list(range(rule.count(n))))
\item ([rule.rank(i) for i in rule.list(n)] == list(range(rule.count(n))))
\item Pour chaque règle d'une grammaire la valeur de la valuation
  calculée, correspond au plus petit mot produit par la règle
\item Si r est une EpsilonRule alors count(n) avec n différent 0
  retourne 0 sinon 1
\item Si r est une SingletonRule alors count(n) avec n différent de
  1 retourne 0 sinon 1
\item Pour toutes les fonctions, pour toutes les règles n négatif
  provoque une exception
\item Pour la fonction count on compare avec les suites connues
  (ajouter ref oeis)
\item Lors du calcul de la valuation on vérifie qu'aucune valuation
  n'a la valeur $\infty$

\end{itemize}

\section {Ajout count, random, list,  unrank}
on implémente les algos de l'énoncé en ajoutant vérification cas < 0

\section {Ajout rank}

Montrer le code\\
Expliquer la modification des paramètres avec valeur = None possible\\
Expliquer comment construire les fonctions avec 2 exemples\\
-> DyckGram\\
-> AB\\

\section{Caching}
Comparaison temps exec avec caching et sans\\
Comparaison temps en fonction des fonctions mémoisées\\
Comparaison mémoisation python2 vs python3 ?\\

\section{Ajout des Grammaires Condensées}
Expliquer traduction\\
Création de nom de règle\\
-> Grammaires condensées dans GrammarsC (pas totalement pour
lisibilité)\\
On execute les test sur les GrammarsC\\
+ Eventuellement comparaison des résultats GrammairesCondensées vs
GrammairesNonCondensées.\\

\section{Constructeur Bound}
Explication\\
- list concat\\
- count sum\\
- unrank et rank\\
Créer tests ?\\

\section {Sequence}
Montrer traduction\\ 
Montrer utilisation pour réécriture de DyckGram\\

\end{document}
